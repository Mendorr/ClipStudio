\section{Introducción}

La creación y edición de vídeo se ha convertido en una demanda clave en múltiples sectores, 
desde educativos y creativos hasta profesionales y empresariales. 
El acceso de forma rápida, sencilla y amigable al usuario final a herramientas digitales 
cada vez es más notorio, permitiendo a usuarios sin formación o conocimientos avanzados en 
el tema poder producir resultados finales de muy buena calidad. 
En este contexto, resulta relevante no solo conocer cómo utilizar dichas herramientas, 
sino también cómo elegirlas y aplicarlas de forma eficiente según objetivos personales 
o de proyecto que se quieran conseguir.


Este documento se centra en el desarrollo de un proyecto práctico centrado 
en la creación de un tráiler de vídeo, con el propósito de analizar y aplicar 
distintas herramientas de edición ampliamente utilizadas en la actualidad, 
sin depender de equipo especializado, como pueda tener la creación de una película o cortometraje. A través de este trabajo, se pretende ofrecer una visión clara y estructurada del proceso de creación y edición de vídeos, sirviendo como guía introductoria para los lectores interesados en este ámbito.

\subsection{Contexto}

La evolución de la edición de vídeo ha ido incrementando el interés sobre sociedad 
en las últimas décadas. Actualmente, acciones como cortar, reorganizar o 
aplicar efectos a secuencias audiovisuales pueden realizarse de forma sencilla 
desde dispositivos tecnológicos. Sin embargo, estos avances actuales contrastan con 
los orígenes de la edición de vídeo, donde el proceso era más complejo y limitado.

Antes de la aparición de máquinas especializadas y programas informáticos, 
la edición de vídeo dependía exclusivamente del criterio del camarógrafo, 
quien debía mantener la toma hasta el momento exacto, sin la existencia de cortes ni de técnicas. 
El desarrollo del software de edición no lineal (NLE) supuso un antes y un después, 
permitiendo trabajar con mayor flexibilidad y creatividad sobre el material grabado.

En la actualidad, tecnologías como el aprendizaje automático y la automatización 
están reformando nuevamente el sector, sustituyendo tareas repetitivas como el 
subtitulado automático, la corrección de color o el enmascaramiento, y ampliando las 
posibilidades creativas de los editores.

\subsection{Problemas a resolver}

La amplia variedad de herramientas disponibles para la creación y edición de vídeos genera, 
en muchos casos, una incertidumbre entre los usuarios a la hora de seleccionar las más adecuadas 
y flexibles según sus necesidades, nivel de experiencia y objetivos creativos.

Este proyecto se centra en solventar dicha problemática mediante el desarrollo 
de un tráiler de vídeo, aplicando las herramientas consideradas por los autores, 
adaptables a sus ideas y creatividad de creación. De este modo, se proporciona una base sólida 
que facilite a otros alumnos la compresión de conceptos fundamentales y los anime 
a indagar herramientas más avanzadas en base a sus preferencias personales.

\subsection{Aportaciones principales}

\begin{enumerate}[label=\arabic*.]
    \item \textbf{Marco metodológico} para la adopción de las herramientas de creación y 
    edición de vídeos, siguiendo buenas prácticas orientas a mejorar la eficiencia de trabajo.
    \item \textbf{Casos de uso prácticos} sobre cada herramienta analizada, 
    facilitando una idea general sobre en qué casos resulta más conveniente 
    utilizar una herramienta u otra.
    \item \textbf{Análisis comparativo} de las ventajas y desventajas del uso de las 
    herramientas explicadas en escenarios reales y profesionales.
    \item \textbf{Guías específicas} que abordan aspectos y conceptos específicos, 
    conversión de formatos y uso de efectos visuales avanzados.
\end{enumerate}


\subsection{Estructura del contenido del documento}

Este documento se estructura con el objetivo de describir de forma progresiva la implementación 
y el uso de herramientas profesionales en la creación y edición de vídeo empleadas en 
el desarrollo del proyecto. De esta manera, se facilita que el lector adquiera los conocimientos 
básicos necesarios y pueda aplicarlos posteriormente en sus propios proyectos o 
entornos laborales.

En primer lugar, se presenta el \textbf{estado del arte}, describiendo los conceptos claves que 
se abordarán a lo largo del presente documento y se analizan herramientas alternativas 
relevantes, así como la situación actual y tendencias futuras en el ámbito de la creación y 
edición de vídeos.

Por otro lado, en el apartado de \textbf{entorno y herramientas}, se describe la metodología de trabajo 
seguida por los autores, detallando el entorno de desarrollo utilizado y las 
herramientas empleadas, junto a su evaluación comparativa respecto a la utilidad 
aportada en el desarrollo del proyecto.

Posteriormente, en el apartado de \textbf{actividades}, se explica el proceso seguido para definir y 
desarrollar la actividad principal del proyecto, proporcionando información sobre 
el flujo de trabajo aplicado y las actividades sistemáticas abordadas para conseguir 
el resultado final.

Asimismo, se incluye un apartado de \textbf{actividades adicionales}, ayudando a reforzar los 
conceptos adquiridos durante la lectura del documento, mediante ejercicios 
prácticos complementarios.

Finalmente, el documento concluye con un apartado de \textbf{conclusiones y referencias}, 
donde cada autor detalla a nivel individual los conocimientos y capacidades adquiridas 
con el desarrollo de este proyecto, además de recopilar referencias donde a modo bibliográfico 
quedan registradas las fuentes externas consultadas para alcanzar el nivel de conocimiento 
que se presenta en el documento.