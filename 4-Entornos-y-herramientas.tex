\section{Entornos y herramientas}
Debido a la evolución exponencial que ha experimentado el marco de la generación y 
edición de vídeo, numerosas empresas han liderado el mercado audiovisual mediante el desarrollo 
de soluciones cada vez más avanzadas. 

Cada una de estas herramientas presenta características diferenciadoras a las propuestas por 
la competencia, lo que ha hecho necesario, para el desarrollo del presente 
proyecto, investigar y probar cada una de ellas de manera independiente. 

Este estudio ha permitido obtener un análisis detallado, haciendo posible la clasificación 
de las herramientas más beneficiosas para alcanzar un resultado final de carácter profesional.

\subsection{Filmora}
\subsubsection{Descripción}

\textit{Filmora} es una aplicación de edición de vídeos nacida en 2010 con su primera aportación 
en el mercado llamada \textit{Video Studio Express}, un generador de vídeo. Con el paso de los años, 
el sistema software fue evolucionando, hasta que se genera la alianza con la empresa 
\textbf{\textit{Wondershare}}, mostrando en 2015 una versión mucho más profesional de la aplicación. 
Esto se debe a la incorporación de funcionalidades impulsadas por IA, sin olvidar la 
mejora en la tasa de fotogramas utilizadas en el editor de la aplicación.\\
Conforme ha ido evolucionando el sector tecnológico, la empresa \textit{Wondershare} 
ha mejorado la calidad de este software, orientándolo actualmente al uso de 
modelos inteligentes (IA) para la generación y edición de vídeos.

\subsubsection{Objetivos y características}

Respecto a su interfaz gráfica, cuenta con un editor de vídeo que ofrece al usuario 
final una gran variedad de funcionalidades útiles en los campos de la edición comercial, 
fotografía, diseño gráfico y marketing, entre otros.\\
Cuenta con una amplia interacción con \textbf{agentes inteligentes}, incorporados 
internamente en el software, ayudando en la experiencia mejorada del usuario. 

La experiencia de uso puede encontrarse en diversos casos condicionada por la 
pertenencia de una subscripción de pago. \\
Este aspecto supone una limitación bastante importante sobre la aplicación de funcionalidades 
dentro de creaciones personales en la plataforma. Además, la aparición progresiva de anuncios en 
ventanas flotantes, puede perjudicar la satisfacción de los usuarios.

\subsubsection{Sistema operativo y requisitos de instalación}

La plataforma de \textit{Filmora} aplica limitaciones de sistema operativo 
para la instalación su instalación local.

Los principales sistemas operativos que soporta son los siguientes:
\begin{itemize}
    \item \textit{Windows}
    \item \textit{MacOS}
    \item \textit{iOS}
    \item \textit{Android}
\end{itemize}

Respecto a las compatibilidades dentro de los sistemas operativos mencionados, 
cabe destacar las siguientes limitaciones:
\textbf{Versiones de sistema operativo}
\begin{itemize}
    \item \textbf{Arquitectura x64}: Windows 11, Windows 10, Windows 8.1, Windows 7.
    \item \textbf{Arquitectura ARM}: Windows 11, Windows 10
\end{itemize}
\textbf{Procesador}

Mínimo recomendado, Intel 6º generación, AMD Ryzen 3 1300X o superior. CPU Intel 9º generación 
o más reciente para reproducción y edición fluida de vídeos HD y 4K.
Para edición de vídeo \textbf{8K} se recomienda un procesador de alta gama como 
Intel Core i9 10ª generación o más reciente, o ARM Ryzen 9 serie 3000 o más reciente 
para una reproducción y edición fluida de vídeo 8K.

\textbf{RAM}

Mínimo recomendado, 8 GB de RAM.16 GB para edición de vídeo HD y 4K.32 GB o más para 
gestionar eficientemente los altos requerimientos de memoria de la edición de vídeo 
en 8K.

\textbf{Gráficos}

Mínimo recomendado, NVDIA GeForce GTX 1050 Ti o posterior; NVIDIA GeForce MX550 
o posterior; AMD Radeon RX 500 o posterior; Intel UHD Graphics 630 o posterior con 
2 GB de VRAM.4 GB de VRAM para la edición de vídeo en HD y 4K. NVIDIA RTX 3000 o la 
serie AMD Radeon RX 6000 con al menos 8 GB de VRAM para la edición de vídeo 8K.

\textbf{Espacio en disco}

Al menos 10 GB de espacio libre en disco duro para la instalación, recomendable 
el uso de discos SSD para trabajar con edición HD 4K. Se recomienda un SSD NVMe rápido 
con al menos 1 TB de espacio de almacenamiento para manejar los grandes tamaños de archivos 
y las tasas de transferencia de datos requeridas para la edición de vídeo en 8K.

\subsubsection{Ventajas frente a otras alternativas}
\begin{itemize}
    \item \textbf{Interfaz y Usabilidad}
    
    Presenta una interfaz sencilla y muy capaz, diseñada para ser intuitiva 
    y fácil de usar. Sus funcionalidades suelen estar orientadas a usuarios 
    inexpertos en el mundo de la edición y creación de vídeo, permitiendo 
    un aprendizaje gradual de los conceptos y conocimientos necesarios a adquirir. 
    Para usuarios con experiencia en este contexto, suelen familiarizarse con la 
    interfaz y funcionalidades de una forma rápida y sencilla.

    \item \textbf{Funcionalidad}
    
    Cuenta con una amplia gama de funciones incluyendo la posibilidad de cortar, agregar efectos 
    prediseñados, múltiples transiciones, efectos, sonidos y más.

    Respecto a la edición en \textit{Filmora}, se siente más tradicional a otros 
    editores y la migración de entorno a uno mucho más profesional reduce su curva 
    de apredizaje.

    De entre las funcionalidades que contiene \textit{Filmora}, destaca respecto 
    a otras herramientas en la disponibilidad de un \textbf{capturador de pantalla} 
    para grabar tutoriales y gameplays.Cuenta con un reencuadrado de vídeo para convertir 
    una secuencia o vídeo de horizontal a vertical.\\
    Posibilita el trabajo con vídeos de 360º y mejora el audio de los vídeos 
    con inteligencia artificial.\\
    Finalmente, cabe destacar la existencia de una función para eliminar y sustituir 
    fondos en la creación y edición de vídeo de forma rápida y sencilla.

    \item \textbf{Aprendizaje}
    
    Una de las claves de \textit{Filmora} en su éxito comercial reside en una sección 
    en su página web dedicada al apredizaje sobre el uso de la herramienta, dispone de 
    numerosos recursos, incluyendo un canal de \textit{YouTube} mostrando y enseñando 
    técnicas y trucos para la edición de vídeo. 

    Por otro lado, \textit{Filmora} dispone de numerosas plantillas y efectos 
    pre diseñados, dando más flexibilidad y usabilidad sobre los usuarios para 
    sus propias creaciones dentro de la herramienta. Este aspecto quizá puede 
    elevar un poco la curva de apredizaje, por lo que es más recomedable hacer 
    uso de ellas si el usuario es experto o tiene conocimientos previos.
\end{itemize}

\subsubsection{Incovenientes frente a otras alternativas}
Pese a que \textit{Filmora} dispone de numerosas ventajas que le hacen como 
software de edición y creación de vídeos, destacar frente a otras alternativas, 
contiene un aspecto importante a considerar que repercute en limitaciones de uso 
y satisfacción del usuario, el \textbf{precio}.

Aunque \textit{Filmora} dispone de un plan gratuito para probar la herramienta 
y sus funcionalidades básicas, si se quiere tener total libertad en el uso de 
sus funcionalidades, dispone un plan de renta mensual y un plan de uso perpétuo 
por 70 USD.\\
La posibilidad de comprar un plan de uso perpétuo y no tener limitaciones de uso 
nunca más dentro de la aplicación, causaba mucha intriga y decisión de compra sobre 
usuarios que se dedican a este sector habitualmente, por trabajo o por proyectos personales.

Cuando \textit{Wondershare} mejoró la versión de \textit{Filmora} pasando de la versión 
11 a la versión 12, numerosos usuarios poseedores del plan perpétuo reportaron 
descontentos con la plataforma al tener que volver a renovar esa subscripción "permanente".\\
Este hecho hizo perder a la empresa mucho público y un descenso de beneficios importante.

\subsubsection{Conclusiones}
Después de analizar en detalle la herramienta por los autores del presente proyecto, 
se llegó a un acuerdo mútuo de ideas, donde se resalta la gran potencia que dispone 
la aplicación para la edición y creación de vídeos, pero genera un rechazo a primera vista, 
tras imponer internamiente la posesión de una subscripción de pago para acceder a funcionalidades 
más potentes que dispone y la posibilidad exportación de vídeos generados. 
Sin olvidar destacar la contínua aparición de \textit{advertisement} 
constante durante el primer contacto con la herramienta.

Por estas razones, se ha decidido no utilizarla como herramienta clave para 
el desarrollo del proyecto.

\subsection{Entorno/Herramienta 1}

\subsection{Entorno/Herramienta X}

\subsection{Conclusiones del análisis de entornos y herramientas}