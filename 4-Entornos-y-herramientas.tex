\section{Entornos y herramientas}
Debido a la evolución exponencial que ha experimentado el marco de la generación y 
edición de vídeo, numerosas empresas han liderado el mercado audiovisual mediante el desarrollo 
de soluciones cada vez más avanzadas. 

Cada una de estas herramientas presenta características diferenciadoras a las propuestas por 
la competencia, lo que ha hecho necesario, para el desarrollo del presente 
proyecto, investigar y probar cada una de ellas de manera independiente. 

Este estudio ha permitido obtener un análisis detallado, haciendo posible la clasificación 
de las herramientas más beneficiosas para alcanzar un resultado final de carácter profesional.

\subsection{Filmora}
\subsubsection{Descripción}

\textit{Filmora} es una aplicación de edición de vídeos nacida en 2010 con su primera aportación 
en el mercado llamada \textit{Video Studio Express}, un generador de vídeo. Con el paso de los años, 
el sistema software fue evolucionando, hasta que se genera la alianza con la empresa 
\textbf{\textit{Wondershare}}, mostrando en 2015 una versión mucho más profesional de la aplicación. 
Esto se debe a la incorporación de funcionalidades impulsadas por IA, sin olvidar la 
mejora en la tasa de fotogramas utilizadas en el editor de la aplicación.\\
Conforme ha ido evolucionando el sector tecnológico, la empresa \textit{Wondershare} 
ha mejorado la calidad de este software, orientándolo actualmente al uso de 
modelos inteligentes (IA) para la generación y edición de vídeos.

\subsubsection{Objetivos y características}

Respecto a su interfaz gráfica, cuenta con un editor de vídeo que ofrece al usuario 
final una gran variedad de funcionalidades útiles en los campos de la edición comercial, 
fotografía, diseño gráfico y marketing, entre otros.\\
Cuenta con una amplia interacción con \textbf{agentes inteligentes}, incorporados 
internamente en el software, ayudando en la experiencia mejorada del usuario. 

La experiencia de uso puede encontrarse en diversos casos condicionada por la 
pertenencia de una subscripción de pago. \\
Este aspecto supone una limitación bastante importante sobre la aplicación de funcionalidades 
dentro de creaciones personales en la plataforma. Además, la aparición progresiva de anuncios en 
ventanas flotantes, puede perjudicar la satisfacción de los usuarios.

\subsubsection{Sistema operativo y requisitos de instalación}

La plataforma de \textit{Filmora} aplica limitaciones de sistema operativo 
para su instalación local.

Los principales sistemas operativos que soporta son los siguientes:
\begin{itemize}
    \item \textit{Windows}
    \item \textit{MacOS}
    \item \textit{iOS}
    \item \textit{Android}
\end{itemize}

Respecto a las compatibilidades dentro de los sistemas operativos mencionados, 
cabe destacar las siguientes limitaciones:

\textbf{Versiones de sistema operativo}
\begin{itemize}
    \item \textbf{Arquitectura x64}: Windows 11, Windows 10, Windows 8.1, Windows 7.
    \item \textbf{Arquitectura ARM}: Windows 11, Windows 10
\end{itemize}
\textbf{Procesador}

Mínimo recomendado, Intel 6º generación, AMD Ryzen 3 1300X o superior. CPU Intel 9º generación 
o más reciente para reproducción y edición fluida de vídeos HD y 4K.
Para edición de vídeo \textbf{8K} se recomienda un procesador de alta gama como 
Intel Core i9 10ª generación o más reciente, o ARM Ryzen 9 serie 3000 o más reciente 
para una reproducción y edición fluida de vídeo 8K.

\textbf{RAM}

Mínimo recomendado, 8 GB de RAM.16 GB para edición de vídeo HD y 4K.32 GB o más para 
gestionar eficientemente los altos requerimientos de memoria de la edición de vídeo 
en 8K.

\textbf{Gráficos}

Mínimo recomendado, NVDIA GeForce GTX 1050 Ti o posterior; NVIDIA GeForce MX550 
o posterior; AMD Radeon RX 500 o posterior; Intel UHD Graphics 630 o posterior con 
2 GB de VRAM.4 GB de VRAM para la edición de vídeo en HD y 4K. NVIDIA RTX 3000 o la 
serie AMD Radeon RX 6000 con al menos 8 GB de VRAM para la edición de vídeo 8K.

\textbf{Espacio en disco}

Al menos 10 GB de espacio libre en disco duro para la instalación, recomendable 
el uso de discos SSD para trabajar con edición HD 4K. Se recomienda un SSD NVMe rápido 
con al menos 1 TB de espacio de almacenamiento para manejar los grandes tamaños de archivos 
y las tasas de transferencia de datos requeridas para la edición de vídeo en 8K.

\subsubsection{Ventajas frente a otras alternativas}
\begin{itemize}
    \item \textbf{Interfaz y Usabilidad}
    
    Presenta una interfaz sencilla y muy capaz, diseñada para ser intuitiva 
    y fácil de usar. Sus funcionalidades suelen estar orientadas a usuarios 
    inexpertos en el mundo de la edición y creación de vídeo, permitiendo 
    un aprendizaje gradual de los conceptos y conocimientos necesarios a adquirir. 
    Para usuarios con experiencia en este contexto, suelen familiarizarse con la 
    interfaz y funcionalidades de una forma rápida y sencilla.

    \item \textbf{Funcionalidad}
    
    Cuenta con una amplia gama de funciones incluyendo la posibilidad de cortar, agregar efectos 
    prediseñados, múltiples transiciones, efectos, sonidos y más.

    Respecto a la edición en \textit{Filmora}, se siente más tradicional a otros 
    editores y la migración de entorno a uno mucho más profesional reduce su curva 
    de apredizaje.

    De entre las funcionalidades que contiene \textit{Filmora}, destaca respecto 
    a otras herramientas en la disponibilidad de un \textbf{capturador de pantalla} 
    para grabar tutoriales y gameplays.Cuenta con un reencuadrado de vídeo para convertir 
    una secuencia o vídeo de horizontal a vertical.\\
    Posibilita el trabajo con vídeos de 360º y mejora el audio de los vídeos 
    con inteligencia artificial.\\
    Finalmente, cabe destacar la existencia de una función para eliminar y sustituir 
    fondos en la creación y edición de vídeo de forma rápida y sencilla.

    \item \textbf{Aprendizaje}
    
    Una de las claves de \textit{Filmora} en su éxito comercial reside en una sección 
    en su página web dedicada al apredizaje sobre el uso de la herramienta, dispone de 
    numerosos recursos, incluyendo un canal de \textit{YouTube} mostrando y enseñando 
    técnicas y trucos para la edición de vídeo. 

    Por otro lado, \textit{Filmora} dispone de numerosas plantillas y efectos 
    pre diseñados, dando más flexibilidad y usabilidad sobre los usuarios para 
    sus propias creaciones dentro de la herramienta. Este aspecto quizá puede 
    elevar un poco la curva de apredizaje, por lo que es más recomedable hacer 
    uso de ellas si el usuario es experto o tiene conocimientos previos.
\end{itemize}

\subsubsection{Incovenientes frente a otras alternativas}
Pese a que \textit{Filmora} dispone de numerosas ventajas que le hacen como 
software de edición y creación de vídeos, destacar frente a otras alternativas, 
contiene un aspecto importante a considerar que repercute en limitaciones de uso 
y satisfacción del usuario, el \textbf{precio}.

Aunque \textit{Filmora} dispone de un plan gratuito para probar la herramienta 
y sus funcionalidades básicas, si se quiere tener total libertad en el uso de 
sus funcionalidades, dispone un plan de renta mensual y un plan de uso perpétuo 
por 70 USD.\\
La posibilidad de comprar un plan de uso perpétuo y no tener limitaciones de uso 
nunca más dentro de la aplicación, causaba mucha intriga y decisión de compra sobre 
usuarios que se dedican a este sector habitualmente, por trabajo o por proyectos personales.

Cuando \textit{Wondershare} mejoró la versión de \textit{Filmora} pasando de la versión 
11 a la versión 12, numerosos usuarios poseedores del plan perpétuo reportaron 
descontentos con la plataforma al tener que volver a renovar esa subscripción "permanente".\\
Este hecho hizo perder a la empresa mucho público y un descenso de beneficios importante.

\subsubsection{Conclusiones}
Después de analizar en detalle la herramienta por los autores del presente proyecto, 
se llegó a un acuerdo mútuo de ideas, donde se resalta la gran potencia que dispone 
la aplicación para la edición y creación de vídeos, pero genera un rechazo a primera vista, 
tras imponer internamiente la posesión de una subscripción de pago para acceder a funcionalidades 
más potentes que dispone y la posibilidad exportación de vídeos generados. 
Sin olvidar destacar la contínua aparición de \textit{advertisement} 
constante durante el primer contacto con la herramienta.

Por estas razones, se ha decidido no utilizarla como herramienta clave para 
el desarrollo del proyecto.

\subsection{CapCut}
\subsubsection{Descripción}
\textit{CapCut} es un editor de vídeo creado por la empresa \textit{Bytedance}, cofundadores de 
la famosa red social \textit{TikTok}, originada para la edición de vídeos cortos de una forma 
sencilla, rápida e intuitiva.

La aplicación fue lanzada por primera vez en China en 2019 y estuvo disponible inicialmente para 
sistemas operativos como \textit{iPhone} y \textit{Android}. En 2020, pasó a llamarse \textit{CapCut} 
y estuvo disponible a nivel internacional.

Actualmente se ha popularizado dentro del marco de creación de contenido en redes sociales gracias 
a sus funcionalidades internas, pudiendo generar resultados muy buenos y de buena calidad, obteniendo 
esa atracción por los consumidores de vídeo.

\subsubsection{Objetivos y características}
Esta herramienta ofrece una amplia gama de funcionalidades para la edición de vídeo, permitiendo que los 
usuarios puedan crear vídeos de alta calidad y compartirlos en las redes sociales.

En relación con la edición de vídeo, los usuarios tienen disponible opciones para cortar y unir 
clips de vídeo, así como agregar efectos de transición entre ellos. A su vez, se puede ajustar la 
velocidad del vídeo, cambiar la orientación, agregar música y silenciar ruido y audio original 
del vídeo grabado.

Respecto a la edición de imágenes, los usuarios pueden ajustar el brillo, el contraste y la 
saturación para mejorar la resolución y resaltar secciones o regiones de interés 
(ROI) dentro de la imagen.

Por otro lado, ofrece la posibilidad de agregar texto y títulos a los vídeos, con una amplia variedad 
de opciones de fuente, tamaño, color y animación. Cabe destacar la posibilidad de generación de subtítulos 
dentro del vídeo, ofreciendo soluciones dentro de la herramienta poco tediosas, para reconocer varios idiomas 
y generar automáticamente subtítulos con bastante precisión.

No hay que olvidar resaltar otras características muy interesantes con las que cuenta esta herramienta, 
como la \textbf{amplia gama de efectos especiales}, existencia de una galería sonora de 
\textbf{música libre de derechos de autor}, la liberta del usuario de \textbf{capturar y grabar la pantalla} 
y la \textbf{exportación y publicación} de vídeos en diferentes formatos y calidades.

\subsubsection{Sistema operativo y requisitos de instalación}
La plataforma \textit{CapCut} se encuentra disponible en cualquier dispositivo móvil.Incluye 
una aplicación especializada para trabajar en PC llamada \textit{CapCut Desktop}. Ambas variantes 
disponen de limitaciones de sistema operativo para su instalación y correcto funcionamiento.

Los principales sistemas operativos que soporta son los siguientes:
\begin{itemize}
    \item \textit{Windows}
    \item \textit{MacOS}
    \item \textit{iOS}
    \item \textit{Android}
\end{itemize}
Esta aplicación está orientada a su uso en dispositivos móviles, por lo que sus requerimientos de 
instalación son más restrictivos que en dispositivos de ordenador.

\textbf{Requisitos de instalación}

\textbf{RAM}

\begin{itemize}
    \item \textit{Android \- Versión reciente}
    
    Para ediciones básicas, es necesario disponer de 2 GB de RAM en el dispositivo. Se aconseja la 
    disponibilidad de 4 o más GB de RAM para un correcto y fluido funcionamiento de la aplicación, 
    especialmente si se comienza a trabajar con multipistas (\textit{multi-track}) y efectos.

    \item \textit{Windows 10 o superior}
    
    Disponibilidad en el dispositivo de al menos 4 GB de RAM.
\end{itemize}

\textbf{Almacenamiento}

\begin{itemize}
    \item \textit{Android \- Versión reciente}
    
    \textit{CapCut} crea caché y ficheros de proyectos. Permite al menos de 4 a 8 GB en almacenamiento 
    libre para ediciones cómodas con exportaciones largas ocasionales. Si se trabaja con proyectos 
    grandes o con mucha resolución, es necesario disponer de más almacenamiento libre que el recomendado.

    \item \textit{Windows 10 o superior}
    
    Disponibilidad en el dispositivo de al menos 2 GB de espacio libre.
\end{itemize}

\textbf{Procesador}

\begin{itemize}
    \item {Android \- Versión reciente}
    
    Es necesario un procesador moderno con al menos 4 cores y una unidad de capacidad gráfica que 
    ayuda con la representación y reproducción de vídeos y animaciones. Es recomendable contar 
    con un apoyo en la aceleración de \textit{Hardware} (GPU) como \textit{OpenGL ES 3.0} o \textit{Vulkan}, 
    estos sportes suelen mejorar la velocidad de exportación de proyectos y la previsualización de los 
    mismos en tiempo real.

    \item {Windows 10 o superior}
    
    Se debe de contar con al menos un procesador Intel Core i3 o superior.
\end{itemize}

\subsubsection{Ventajas frente a otras alternativas}

\begin{itemize}
    \item \textbf{Interfaz y Usabilidad}
    
    Presenta una interfaz sencilla y muy capaz, diseñada para ser intuitiva y fácil de usar. 
    La capacidad de uso está orientada hacia usuarios inexpertos en su marco contextual de 
    edición de vídeos, por lo que su curva de aprendizaje no es abrumadora hacia nuevos usuarios. 
    Por otro lado, usuarios con experiencia previa pueden adaptarse con facilidad y rapidez a la herramienta 
    y a su funcionamiento interno.

    Una ventaja que incluye en su interfaz es su adaptabilidad al dispositivo móvil donde se utiliza, 
    por lo que se encuentra comodidad de uso.

    \item \textbf{Funcionalidad}
    
    Cuenta con una amplia gama de funciones incluyendo la posibilidad de cortar, agregar efectos prediseñados, 
    múltiples transiciones, efectos de texto, biblioteca musical, sonidos y más.

    La forma de editar que dispone esta herramienta es muy intuitiva en su versión móvil, pero 
    al usar su versión de escritorio en un ordenador suele perder ese sentido de uso.

    \textit{CapCut} destaca con creces a otras herramientas similares en el mercado audiovisual, gracias 
    a la disponibilidad de una gran cantidad de efectos y transiciones pre diseñadas, \textit{stickers} 
    o pegatinas, efectos de audio y vídeo con una librería musical sin derechos de autor bastante 
    extensa. Además, incluye la posibilidad de enlazar la cuenta registrada en la herramienta con 
    plataformas de redes sociales para utilizar audios o plantillas de vídeo que se guardaron allí.

    Incorpora efectos muy buenos que se pueden aplicar de una forma muy sencilla, simplemente 
    realizando el proceso de \textbf{presionar y arrastrar} hacia el clip, tomando un tiempo más 
    considerable de hacer en otras herramientas más profesionales.

    Por último, es capaz de suavizar y aclarar rostros, permitiendo resaltar zonas de la cara, obteniendo 
    resultados más profesionales y realistas. Sin olvidar, la generación automática de subtítulos, 
    funcionalidad que otras herramientas similares no incorporan.

    \item \textbf{Aprendizaje}
    
    Aunque no existe documentación exhaustiva ni detallada de CapCut, se encuentran disponibles 
    numerosos vídeos en la plataforma \textit{YouTube}, encargados de proporcionarte conocimientos 
    a cerca de las creaciones que quieras realizar dentro de la aplicación. \\
    Este aspecto permite un aprendizaje interno sobre las funcionalidades que incorpora la herramienta 
    mucho más rápido que tener que leer cientos de secciones en una documentación oficial.

    Por otro lado, \textit{CapCut} suele ofrecer cursos gratis de formación que ayudan a entender 
    mejor el funcionamiento de la herramienta.
\end{itemize}

\subsubsection{Inconvenientes frente a otras alternativas}
Después de desengranar la herramienta completa, \textit{CapCut} parece que no presenta inconvenientes, 
es amigable al usuario, sencillo de utilizar, disponible en muchos sistemas operativos, no incluye 
anuncios ni limitaciones por subscripciones de pago y demás. 

Muchos usuarios que han experimentado con esta herramienta encuentran un inconveniente que puede 
ocurrir de manera sobrevenida y es la incorporación de cuotas mensuales por uso y restricciones 
económicas que muchas otras herramientas competidoras ya incorporan.\\
Por otro lado, su empresa creadora no deja de ser \textit{ByteDance}, cofundadora de la famosa 
red social \textit{TikTok}, por sucesos experimentados mundialmente, pueden ocurrir también problemas 
relacionados con las políticas de privacidad y la información personal de usuario dentro de esta herramienta. 

\subsubsection{Conclusiones}
\textit{CapCut} ha demostrado ser una de las herramientas más completas dentro del marco audiovisual 
actual. Destaca por su flujo de trabajo intuitivo y su facilidad de uso, lo que la convierte en una 
opción atractiva para la creación de contenido.

Tras el análisis realizado por los autores del presente proyecto, se identificaron numerosas ventajas 
que podrían resultar útiles en el desarrollo del tráiler. No obstante, se ha descartado su 
utilización como herramienta principal, dado que está orientada principalmente a la edición de vídeo 
en formato móvil, lo cual no se ajusta al enfoque del proyecto.

Por estas razones, se ha decidido no utilizar esta herramienta como parte clave en el desarrollo del proyecto.

\subsection{Canva}
\subsubsection{Descripción}

Canva es una plataforma de diseño gráfico online, muy intuitiva y accesible, que permite a cualquier 
persona crear gráficos, presentaciones, carteles, documentos y otros contenidos visuales sin tener 
experiencia previa en diseño.

Esta herramienta se basa en la filosofía del diseño \textit{drag-and-drop} (arrastrar y soltar), 
lo que significa que se puede añadir y organizar elementos visuales fácilmente en el lienzo.

Desde su lanzamiento en 2013, se ha convertido en una solución esencial para marketing, emprendedores 
y estudiantes.

Recientemente, la plataforma ha registrado en el mercado una solución muy interesante dentro del 
marco de contexto en el proyecto, denominada \textit{Canva Video}, una herramienta que sirve de 
editor de vídeo completo, ofreciendo flexibilidad y control al alcance de todo tipo de usuarios.

\subsubsection{Objetivos y características}
Desde su lanzamiento al mercado audiovisual, \textit{Canva} siempre ha ido un paso por delante 
respecto a sus empresas competidoras, el enfoque siempre ha estado en la mejora contínua de sus 
productos ofrecidos y la adaptación constante a los cambios que se producen en el sector tecnológico y 
obteniendo beneficio mútuo con sus clientes.

El principal objetivo de \textit{Canva} siempre ha sido la democratización de los sectores en los que 
se mueve y ofrece productos. El mundo de la edición de vídeo era cuanto menos interesante y por eso 
lanzaron su versión de vídeo \textit{Canva Video}, buscando ofrecer características a través de su 
plataforma muy únicas y útiles, para atraer a todo tipo de clientes.

Uno de sus principales objetivos fue la \textbf{simplificación de procesos}, 
con la creación de un editor de diseño sencillo y aplicarlo al formato de vídeo, integrando 
plantillas, música y animaciones en una sola experiencia intuitiva.

Por otro lado, \textit{Canva} busca posicionarse como una plataforma integral de comunicación 
visual, permitiendo a los usuarios crear, editar, grabar y colaborar en vídeos en la nube sin 
tener que cambiar de aplicación.

Finalmente, con sus versiones más recientes, intentan adaptarse como cualquier otra herramienta 
competente al auge de los \textbf{agentes inteligentes}, permitiendo generar un vídeo completo 
a través de un simple mensaje de texto.

\subsubsection{Sistema operativo y requisitos de instalación}
Una de las características relacionadas con las especificaciones técnicas que dispone \textit{Canva} 
es la posibilidad de trabajar con su herramienta \textit{Canva Video} en navegadores, por lo que 
no es necesario tener su versión en un entorno local.

Sin embargo, si se encuentra más cómodo el usuario trabajando en entorno local, se indican 
ciertos requerimientos básicos para su instalación.

Los principales sistemas operativos que soporta son los siguientes:
\begin{itemize}
    \item Windows
    \item MacOS
    \item iPad
    \item iPhone
    \item Android
\end{itemize}

Las versiones de navegador que soporta son las siguientes:
\begin{itemize}
    \item \textbf{Google Chrome}, versión 84 o superior
    \item \textbf{Mozilla Firefox}, versión 91 o superior
    \item \textbf{Safari}, versión 14.1 o superior
    \item \textbf{Microsoft Edge}, versión 84 o superior
    \item \textbf{Opera GX}, versión 74 o superior
\end{itemize}

Respecto a los requerimientos básicos sobre los sistemas operativos que soporta, se indican 
las siguientes recomendaciones:

\textbf{Versiones móviles}

Se requiere una versión del sistema operativo \textit{iOS} de 14.5 o superior. Se debe disponer 
de 50 MB de momeoria libre como mínimo.

\textbf{Versiones de escritorio}

\begin{itemize}
    \item \textbf{CPU}
    
    Para \textit{Windows} se recomienda como mínimo 1 GHz (dual-core) o procesadores más rápdiso.\\
    Para \textit{MacOS} se recomiendan 64-bit \textit{Intel or Apple M1}

    \item \textbf{RAM}
    
    Para \textit{Windows} se recomiendam 4 GB.\\
    Para \textit{MacOS} se recomiendan 4 GB.

    \item \textbf{Espacio}
    
    Para \textit{Windows} se necesita 1 GB.\\
    Para \textit{MacOS} se necesita 1 GB.
\end{itemize}

\subsubsection{Ventajas frente a otras alternativas}

\begin{itemize}
    \item \textbf{Interfaz y Usabilidad}
    
    \textit{Canva Video} se encuentra integrada dentro de la platafoma principal \textit{Canva}, 
    manteniendo un estilo sencillo y aplicando la técnica comúnmente utilizada, \textit{drag-and-drop} 
    (arrastrar y soltar), facilitando que cualquier persona sin experiencia previa en edición de vídeo 
    pueda hacer uso de ella sin mucha complejidad.

    El diseño de la interfaz presenta una estructura bastante intuitiva y peculiar. En la parte lateral 
    izquierda, se encuentran las herramientas que se pueden aplicar a la edición de vídeo, 
    son presentados en formato objeto para poder arrastrarlos hacia el vídeo fácilmente.\\
    En la zona central de la interfaz, se dispone el lienzo de trabajo donde se visualiza el vídeo 
    importado y se puede ir previsualizando la edición del mismo.\\
    Para concluir, en la parte inferior de la interfaz, se presenta una línea temporal para 
    organizar clips y ajustar duraciones.

    Pasando a la usabilidad de la herramienta, presenta una curva de aprendizaje muy leve, permitiendo 
    empezar con una plantilla en la que basar el trabajo, pudiendo editarlo a gusto del usuario en 
    cuestión de minutos.\\
    Contiene numerosos recursos gratuitos (p.ej. plantillas, clips, música) que reducen el tiempo 
    de creación. Este aspecto es muy importante para permitir que los usuarios consigan resultados 
    bastante buenos y profesionales, pudiendo obtener un plus de recursos a través de membresías de pago.

    Al formar parte de \textit{Canva}, se pueden reutilizar diseños gráficos, de imágenes o elementos 
    creados por otras publicaciones o redes sociales.
    \item \textbf{Funcionalidad}
    
    Cuenta con una amplia variedad de funcionalidades que consiguen generar en los usuarios 
    una capacidad y comodidad de uso bastante aceptable.

    Ofrece cientos de plantillas profesionales organizadas por tipo (intros, promocionales, etc.).\\
    Asimismo, permite previsualizar las plantillas antes de importarlas al lienzo de trabajo, de esta 
    manera el usuario se puede hacer una idea de cómo puede enfocar su creación si incluye ciertas 
    plantillas, evitando "\textit{manchar}" el lienzo de trabajo con contenido que al final no se 
    utilizará.

    Permite recortar vídeos, dividirlos y mover elementos del mismo para estructurar el vídeo a gusto 
    del usuario. Con la zona inferior de la interfaz, dedicada a la edición temporal del vídeo, como 
    ya se ha comentado en el anterior punto, otorga la posibilidad de arrastrar y soltar clips de vídeo, 
    imágenes, audios y textos directamente en una línea de tiempo visual.

    Una de las funcionalidades más importantes que incorpora esta herramienta reside en su biblioteca 
    interna de audio, pudiendo almacenar música de fondo, efectos de sonido, listos para usar en cualquier 
    proyecto a realizar, permitiendo poder \textbf{reutilizar} música y efectos de sonido utilizados 
    en anteriores creaciones, configurados con dedicación y teniendo una complejidad importante.\\
    Por otro lado, permite ajustar el volumen, cortar audios e implementar filtrados de entrada/salida en el audio.

    Como en cualquier otra herramienta competidora dentro de este mercado audiovisual, esta plataforma 
    permite incluir transiciones sencillas entre los clips del vídeo. Unas de las habituales y se aplican 
    en numerosos proyectos son los fundidos, deslizamientos, zoom, entre otras.
    \item \textbf{Aprendizaje}
    
    La plataforma principal \textit{Canva} ofrece documentación bastante detallada sobre el uso de 
    \textit{Canva Video}, tanto para usuarios recién iniciados en el mundo de la creación y edición de 
    vídeos, como de usuarios expertos que necesitan habituarse en un corto periodo de tiempo con su 
    interfaz de trabajo.

    Cuenta con diversos apartados, relacionados con el diseño, creación y edición de vídeo, además 
    de ofrecer documentación adicional para previsualizar el progreso del proyecto en desarrollo, 
    tipos de exportaciones que soporta y demás. 

    Cabe destacar lo cuidada que se encuentra su documentación, aplicando explicaciones para las 
    dos versiones más demandadas de su plataforma: versión móvil y versión de escritorio.
\end{itemize}

\subsubsection{Inconvenientes frente a otras alternativas}
A pesar de que \textit{Canva Video} presenta numerosas ventajas que imponen su ranking (posicionamiento) 
entre las herramientas más demandadas actualmente en el mercado audiovisual, presenta ciertas desventajas 
que los usuarios han ido experimentando y reportando con su uso.

\begin{itemize}
    \item \textbf{Limitaciones en personalización avanzada}
    
    Aunque \textit{Canva Video} es muy flexible, tiene limitaciones en comparación con programas de 
    diseño profesional. No permite trabajar con vectores complejos ni ofrece herramientas avanzadas 
    necesarias para ciertos proyectos de diseño.

    \item \textbf{Calidad de creaciones}
    
    Si bien los diseños y plantillas de \textit{Canva Video} son ideales para su uso digital, en ocasiones 
    la calidad puede no ser suficiente para proyectos que requieren una resolución extremadamente alta.

    \item \textbf{Dependencia de internet para su uso}
    
    La mayoría de funciones que ofrece \textit{Canva Video} requieren conexión a \textit{internet}. 
    Esto puede ser una desventaja si se necesita trabajar en lugares con acceso limitado o sin conexión.
\end{itemize}

\subsubsection{Conclusiones}

Al concluir el análisis exhaustivo de esta herramienta por los autores del presente proyecto, se 
identificaron numerosas ventajas que podrían resultar útiles en el desarrollo del tráiler de vídeo. 
Se han tenido en cuenta los inconvenientes que presentan, pero no se quiere conseguir con el desarrollo 
del tráiler un producto asemejado con uno cinemático, es puramente educativo y no se demanda tanto 
la alta resolución del resultado.

Por estas razones, no se descarta el uso de esta herramienta como parte clave en el desarrollo del proyecto.

\subsection{Adobe Alter Effects}

\subsection{Conclusiones del análisis de entornos y herramientas}