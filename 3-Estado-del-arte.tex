\section{Estado del arte}

El estado del arte es una etapa fundamental de cualquier investigación o proyecto. En el ámbito del presente trabajo abarca el conjunto de competencias principales que los autores consideran esenciales para alcanzar los objetivos establecidos. Así, la información reflejada en este apartado debe evidenciar un estudio previo por parte de los autores sobre los principales conceptos que deben conocer para desarrollar adecuadamente el proyecto.

Por ejemplo, en un proyecto sobre técnicas de diseño y animación 3D convendría explicar conceptos como \textit{box modeling}, \textit{motion-design} o \textit{stop-motion}, mientras que uno centrado en la edición de vídeos abordaría otros como, por ejemplo, los píxeles, las resoluciones, los tipos de formatos de vídeo actuales, o las relaciones de aspecto más habituales, entre otros.

El contenido de esta sección deberá incluir las correspondientes citas bibliográficas sobre las fuentes de información empleadas para su redacción.
