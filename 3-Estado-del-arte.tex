\section{Estado del arte}

La creación y edición de vídeo constituye uno de los pilares fundamentales dentro del ámbito del Procesamiento de la Información Multimedia, ya que combina aspectos técnicos, artísticos y comunicativos. El desarrollo de un proyecto audiovisual eficaz requiere un conocimiento previo de los conceptos básicos que intervienen en la captura, tratamiento, codificación y distribución del contenido visual en movimiento. 

\subsection{Conceptos básicos}

Para hablar de la edición de vídeo digital debemos de tener claros algunos de sus elementos esenciales. Como el \textbf{píxel}, la unidad mínima de información visual de la que se compone una imagen. La disposición y cantidad de píxeles determinan la \textbf{resolución}, un factor que determina la calidad del vídeo. Entre las resoluciones más habituales se encuentran HD (1280×720), Full HD (1920×1080) y 4K (3840×2160), ampliamente utilizadas en distintos contextos audiovisuales según las necesidades de calidad y rendimiento (Jack, 2019). 

Otros conceptos fundamentales son la \textbf{relación de aspecto}, que define la proporción entre el ancho y el alto de la imagen, 16:9, estándar en televisión y plataformas digitales, y 21:9 en producciones cinematográficas. O la \textbf{fluidez del movimiento}, el número de fotogramas por segundo (FPS). Valores como 24 FPS se asocian tradicionalmente al cine, mientras que 30 o 60 FPS son habituales en contenidos televisivos y digitales, especialmente en videojuegos o retransmisiones en directo. Normalmente, cuanto mayor sea la resolución y tasa de fotogramas de un vídeo, mayor será la calidad de este, así como el tamaño de este.

\begin{figure}[H]
    \centering
    \includegraphics[width=0.5\linewidth]{assets/Resoluciones.png}
    \caption{Resoluciones comunes}
    \label{fig:placeholder}
\end{figure}

 En la anterior figura se pueden observar las resoluciones más comunes. Desde 360p que en su día era considerado como HD (High Definition) hasta resoluciones más modernas como el 4K o el 8k, las cuales no son visibles en todas las pantallas.

 \subsubsection{Postproducción}
 Por otro lado, la \textbf{postproducción} es una de las fases más determinantes en la creación y edición de vídeo, ya que en ella se da forma definitiva al material audiovisual grabado. Las posibilidades de la edición son infinitas y van únicamente sujetas a la creatividad y conocimientos del editor. Uno de los principales objetivos es conseguir obtener una narrativa mezclando las distintas vistas (Dancyger, 2014). La postproducción incluye también la incorporación de efectos visuales (VFX), como transiciones, composiciones digitales, títulos o animaciones, que amplían las posibilidades creativas del medio y facilitan la integración de elementos gráficos en el vídeo final (Watkinson, 2018).

 Pero antes de profundizar en este mundo, es necesario quedar claros algunos conceptos básicos. Una serie de operaciones básicas que permiten organizar, estructurar y perfeccionar el material grabado hasta obtener una pieza coherente y finalizada. Entre las acciones más elementales se encuentran cortar y dividir clips, lo que permite eliminar fragmentos innecesarios y ajustar la duración de cada plano para mejorar el ritmo narrativo. Del mismo modo, la operación de copiar y pegar facilita la reorganización del material dentro de la línea de tiempo, permitiendo modificar el orden de las escenas o reutilizar fragmentos concretos. Estos procesos se realizan dentro de un entorno de edición no lineal (NLE), donde el contenido se organiza en pistas de vídeo y pistas de audio independientes.
 \begin{figure} [H]
    \centering
    \includegraphics[width=0.5\linewidth]{assets/editor.jpg}
    \caption{Editor de vídeo}
    \label{fig:placeholder}
\end{figure}
 
 Las pistas de vídeo permiten superponer imágenes, añadir títulos o incorporar recursos gráficos, mientras que las pistas de audio posibilitan separar diálogos, música y efectos sonoros para tratarlos de forma individual. Esta estructura por capas proporciona un mayor control técnico y creativo, ya que permite ajustar sincronización, volumen, transiciones y efectos sin alterar el resto del proyecto. En conjunto, estos conceptos básicos constituyen la base operativa sobre la que se construyen técnicas más avanzadas de edición y efectos visuales dentro del flujo de trabajo de postproducción.

Las \textbf{máscaras} son un concepto fundamental en la postproducción de vídeo, ya que de su uso dependen una gran cantidad de técnicas. Son herramientas que permiten delimitar una zona específica del encuadre para aplicar sobre ella un efecto, corrección o modificación sin afectar al resto de la imagen. Técnicamente, una máscara define una región mediante una forma geométrica o un trazado personalizado que puede ajustarse manualmente y animarse a lo largo del tiempo mediante fotogramas clave, lo que posibilita que dicha región siga el movimiento de un objeto dentro del plano. Su uso es fundamental para procesos de composición digital, corrección de color selectiva y efectos visuales. Por ejemplo, permiten realizar una congelación parcial del vídeo (manteniendo inmóvil un sujeto mientras el entorno continúa en movimiento), como podemos ver en la siguiente figura:
\begin{figure}[H]
    \centering
    \includegraphics[width=0.5\linewidth]{assets/maxresdefault.jpg}
    \caption{Efecto de congelación parcial -Reddit Filmakers}
    \label{fig:placeholder}
\end{figure}

Aplicar slow motion selectivo únicamente a una parte del encuadre, ocultar elementos no deseados, integrar gráficos o textos en áreas concretas o incluso llevar a cabo efectos de clonación, combinando varias tomas y ocultando las zonas sobrantes para simular la presencia múltiple de un mismo sujeto en un mismo plano. 
\begin{figure}[H]
    \centering
    \includegraphics[width=0.5\linewidth]{assets/Clonacion.jpg}
    \caption{Efecto clonación - Youtube:Carlos Lang}
    \label{fig:placeholder}
\end{figure}
Asimismo, las máscaras son esenciales en técnicas más avanzadas como el seguimiento de movimiento (motion tracking) y la composición por capas, donde el control preciso de las regiones de la imagen resulta clave para obtener resultados realistas.


 Como rencien comentábamos, las máscaras influyen en el etalonaje y la corrección de \textbf{color}. Estas constituyen fases fundamentales dentro de la postproducción audiovisual. Responden a una necesidad técnica de la imagen como a la construcción estética y narrativa del producto final. El etalonaje se trata del proceso creativo en el que se aplican ajustes de color en una imagen o vídeo para hacer más atractiva y mejorar la calidad visual de la producción.

Los creadores conceden una gran dedicación al etalonaje ya que con esta técnica se pueden añadir elementos dramáticos o enfatizar estados de ánimo a una obra. La evolución del etalonaje, también conocido como grading, ha sido posible gracias a herramientas digitales que han permitido que sea más asequible para cualquier tipo de producción, y ha definido notablemente el estilo de las películas de los últimos años.
\begin{figure}[H]
     \centering
     \includegraphics[width=0.5\linewidth]{assets/etalonaje-y-correcion-de-color-blog-ejemplos.jpg}
     \caption{Etalonaje del color - La La Land (2016)}
     \label{fig:placeholder}
 \end{figure}

 La corrección de color es el procedimiento de ajustar la reproducción de los colores en una imagen o vídeo para que sean precisos y coherentes con un estándar o referencia concreta.
 Este proceso se realiza para lograr una continuidad visual. Por ejemplo, cuando se filma una escena, cada plano puede ser grabado en diferentes momentos del día, con diferentes condiciones de iluminación y enfoque, lo que puede hacer que los colores parezcan diferentes de un plano a otro.(Lourdes Millán - Animun3D)
   \begin{figure}[H]
     \centering
     \includegraphics[width=0.5\linewidth]{assets/Color.jpg}
     \caption{Corrección del color - La La Land (2016)}
     \label{fig:placeholder}
 \end{figure}
 
 Mientras que la corrección de color se centra en equilibrar parámetros como la exposición, el contraste, la temperatura de color y la saturación para garantizar coherencia visual entre planos, el etalonaje persigue una intención expresiva, definiendo atmósferas, estilos y emociones mediante el tratamiento cromático. En la actualidad, estas técnicas han evolucionado significativamente gracias al desarrollo de herramientas digitales avanzadas que permiten un control preciso sobre curvas, nodos, máscaras y espacios de color, consolidándose como un elemento clave en producciones cinematográficas, televisivas y de contenido digital.

 Como ya hemos mencionado, \textbf{los efectos visuales (VFX)} constituyen uno de los pilares fundamentales de la postproducción audiovisual contemporánea. Ya que permiten la creación, modificación o integración de elementos visuales que no pueden capturarse directamente durante el rodaje, evitando así, situaciones peligrosas o costes innecesarios. Los VFX combinan todo tipo de técnicas de procesamiento digital de la imagen, composición por capas, animación y simulación. Se utilizan tanto en intervenciones sutiles (como la eliminación de objetos no deseados) hasta la creación de entornos completamente digitales, criaturas generadas por ordenador o simulaciones físicas complejas.

 Uno de los efectos visuales más comunes es el \textbf{ChromaKey}, donde el el fondo y el primer término de la imagen han sido rodados de forma independiente. Este proceso no es nada nuevo y lo que ha cambiado respecto a su forma tradicional ha sido que ahora se ruedan en analógico o digital.
 Pero como podemos ver en la figura 7 (Mary Poppins, donde se utilizo el sistema de croma con lámparas de vapor de sodio.), es una técnica utilizada desde los inicios del cine.)
\begin{figure}[H]
     \centering
     \includegraphics[width=0.5\linewidth]{assets/marypoppins.jpg}
     \caption{Set de grabación de Mary Poppins}
     \label{fig:placeholder}
 \end{figure}

 Actualmente es mucho más común el empleo de la \textit{pantalla verde o azul}, donde un fondo de color uniforme, facilita la separación digital entre el sujeto grabado y el entorno que posteriormente se añadirá en postproducción. La elección del verde o del azul no es arbitraria: ambos colores se encuentran alejados de los tonos habituales de la piel humana, lo que permite aislar con mayor precisión la figura del sujeto mediante procesos de eliminación selectiva de color. En la práctica, el verde suele emplearse con mayor frecuencia en entornos digitales actuales debido a su mayor luminosidad y mejor respuesta en sensores de cámaras digitales, mientras que el azul fue históricamente más común en procesos fotoquímicos tradicionales.
 
 \begin{figure}[H]
     \centering
     \includegraphics[width=0.5\linewidth]{assets/fuego.jpg}
     \caption{Ejemplo de efecto fuego en greenscreen}
     \label{fig:placeholder}
 \end{figure}

 Es muy común utilizar efectos como el que se ve en la anterior figura, donde la parte que queremos implementar en nuestro metraje (en este caso el fuego) aparece sobre un fondo verde. La mayoría de programas de edición de vídeo tienen alguna herramienta que permite eliminar la pantalla verde y que quede únicamente el efecto deseado. Este ejemplo es sencillo, pero es aplicable a efectos de mayor escala y complejidad.
 
 Actualmente, y como evolución de las pantallas verdes tradicionales, existen los LED Volume(también conocidos como virtual production stages), Consisten en grandes estructuras semicirculares o envolventes formadas por pantallas LED de alta resolución sobre las que se proyectan entornos digitales generados en tiempo real mediante motores gráficos. A diferencia del chroma key tradicional, donde el fondo se sustituye posteriormente en postproducción, en los sistemas LED Volume el entorno virtual se visualiza directamente durante el rodaje, lo que permite que la iluminación y los reflejos del escenario digital afecten de manera natural a los actores y objetos físicos presentes en el set. Se ha popularizado por su uso en los últimos proyectos de Star Wars.
\begin{figure}[H]
     \centering
     \includegraphics[width=0.5\linewidth]{assets/ledvolumes.jpg}
     \caption{Set de grabación The Mandalorian}
     \label{fig:placeholder}
 \end{figure}

Tambien popular en esta clase de producciones es el uso del \textbf{CGI (Computer Generated Imagery)},  hace referencia a la creación de imágenes, entornos o elementos visuales, pero esta vez íntegramente por ordenador. A diferencia de la animación tradicional, el CGI se utiliza con frecuencia como complemento dentro de producciones de imagen real, integrando objetos, criaturas o escenarios digitales en grabaciones físicas. Por otra parte, el CGI también se diferencia del uso de pantalla verde (greenscreen), ya que esta última no genera por sí misma los elementos digitales; simplemente facilita su integración posterior. En otras palabras, el greenscreen es una herramienta de aislamiento y composición, mientras que el CGI es el proceso de creación digital de los elementos que pueden sustituir o complementar la imagen real. En el cine contemporáneo, ambas técnicas suelen combinarse para lograr escenas híbridas donde actores reales interactúan con entornos y personajes generados por ordenador, ampliando así las posibilidades narrativas y visuales del medio audiovisual. Se puede apreciar en la siguiente figura:
\begin{figure}[H]
    \centering
    \includegraphics[width=0.5\linewidth]{assets/cgi.jpg}
    \caption{Enter Caption}
    \label{fig:placeholder}
\end{figure}
 Y a día de hoy es imposible hablar sobre creación de vídeo sin mencionar a las IA generativas. Capaces de transformar unas pocas líneas de texto en clips profesionales, ahorrando tiempo y costes en producción. A su vez crece una fuerte crítica sobre el uso de esta, ya que se basa en patrones, datos previos e incluso en haberla entrenado con contenidos ya existentes. Esto hace que rara vez aporte un enfoque único o creativo. Un vídeo con IA puede cumplir con la estructura básica, pero difícilmente transmitirá la esencia de una marca, su personalidad o sus valores diferenciales. Entrando aún más en la ética de usar la IA, es el impacto que tiene en el empleo dentro del sector audiovisual. La creatividad humana no solo aporta un resultado estético, también genera trabajo en toda una industria cultural. Usar IA de forma indiscriminada va hacer que se acaben desvalorizando estos oficios.

 En conclusión, la creación y edición de vídeo constituyen un campo interdisciplinar que integra fundamentos técnicos del procesamiento digital de la imagen y el sonido con toda clase de fines. A lo largo del desarrollo tecnológico, herramientas como el montaje no lineal, el CGI, el chroma key o los sistemas de producción virtual han ampliado de forma significativa las posibilidades creativas del medio. No obstante, este ámbito esta sujeto a las limitaciones técnicas y operativas, tales como la alta demanda de recursos computacionales, la necesidad de hardware especializado, la complejidad de los flujos de trabajo y la constante actualización tecnológica que exige formación continua. Asimismo, la evolución de la producción audiovisual implica la aparición de múltiples perfiles profesionales especializados —editores, coloristas, técnicos de sonido, artistas de VFX, animadores 3D o supervisores de postproducción— que trabajan de forma coordinada en entornos colaborativos. Y con la llegada de la IA, se abren muchas posibilidades y cambios por llegar a este sector. En este contexto, el conocimiento de los fundamentos teóricos y técnicos analizados en el presente estado del arte resulta esencial para comprender los procesos implicados en la creación de contenido audiovisual y para abordar con rigor el desarrollo del proyecto planteado.

\subsection{Fuentes}
\begin{itemize}

\item Brown, B. (2016). \textit{Cinematography: Theory and Practice: Image \item Making for Cinematographers and Directors. Focal Press. }
\item Dancyger, K. (2014). \textit{The Technique of Film and Video Editing: \item History, Theory, and Practice. Focal Press.}
\item Jack, K. (2019). \textit{Video Demystified: A Handbook for the Digital Engineer. Newnes. }
\item Watkinson, J. (2018). \textit{The Art of Digital Video. Focal Press. }
\end{itemize}



