\section{Estado del arte}

La creación y edición de vídeo constituye uno de los pilares fundamentales dentro del ámbito del Procesamiento de la Información Multimedia, ya que combina aspectos técnicos, artísticos y comunicativos. El desarrollo de un proyecto audiovisual eficaz requiere un conocimiento previo de los conceptos básicos que intervienen en la captura, tratamiento, codificación y distribución del contenido visual en movimiento. 

\subsection{Conceptos básicos}

Para hablar de la edición de vídeo digital debemos de tener claros algunos de sus elementos esenciales. Como el \textbf{píxel}, la unidad mínima de información visual de la que se compone una imagen. La disposición y cantidad de píxeles determinan la \textbf{resolución}, un factor que determina la calidad del vídeo. Entre las resoluciones más habituales se encuentran HD (1280×720), Full HD (1920×1080) y 4K (3840×2160), ampliamente utilizadas en distintos contextos audiovisuales según las necesidades de calidad y rendimiento (Jack, 2019). 

Otros conceptos fundamentales son la \textbf{relación de aspecto}, que define la proporción entre el ancho y el alto de la imagen, 16:9, estándar en televisión y plataformas digitales, y 21:9 en producciones cinematográficas. O la \textbf{fluidez del movimiento}, el número de fotogramas por segundo (FPS). Valores como 24 FPS se asocian tradicionalmente al cine, mientras que 30 o 60 FPS son habituales en contenidos televisivos y digitales, especialmente en videojuegos o retransmisiones en directo. Normalmente, cuanto mayor sea la resolución y tasa de fotogramas de un vídeo, mayor será la calidad de este, así como el tamaño de este.  

Por otro lado, la \textbf{postproducción} es una de las fases más determinantes en la creación y edición de vídeo, ya que en ella se da forma definitiva al material audiovisual grabado. Las posibilidades de la edición son infinitas y van únicamente sujetas a la creatividad y conocimientos del editor. Uno de los principales objetivos es conseguir obtener una narrativa mezclando las distintas vistas (Dancyger, 2014). La postproducción incluye también la incorporación de efectos visuales (VFX), como transiciones, composiciones digitales, títulos o animaciones, que amplían las posibilidades creativas del medio y facilitan la integración de elementos gráficos en el vídeo final (Watkinson, 2018).

El etalonaje y la corrección de \textbf{color} constituyen fases fundamentales dentro de la postproducción audiovisual, orientadas tanto a la corrección técnica de la imagen como a la construcción estética y narrativa del producto final. Mientras que la corrección de color se centra en equilibrar parámetros como la exposición, el contraste, la temperatura de color y la saturación para garantizar coherencia visual entre planos, el etalonaje persigue una intención expresiva, definiendo atmósferas, estilos y emociones mediante el tratamiento cromático. En la actualidad, estas técnicas han evolucionado significativamente gracias al desarrollo de herramientas digitales avanzadas que permiten un control preciso sobre curvas, nodos, máscaras y espacios de color, consolidándose como un elemento clave en producciones cinematográficas, televisivas y de contenido digital.
 
Paralelamente, el tratamiento del \textbf{audio}, que engloba procesos como la eliminación de ruido, la ecualización, la mezcla de pistas y la sincronización con la imagen, resulta esencial para garantizar una experiencia audiovisual inmersiva y profesional, siendo un componente clave en la percepción de calidad por parte del espectador. En conjunto, la postproducción actúa como un proceso integrador en el que convergen aspectos técnicos y artísticos, permitiendo transformar el material audiovisual bruto en un producto final que cumpla con los estándares de calidad exigidos en el ámbito del procesamiento multimedia. 

\subsection{Investigaciones recientes}
En el ámbito académico y de investigación, cabe destacar ciertos proyectos en desarrollo que 
generan día a día una nueva manera de pensar y desarrollar herramientas para la creación y edición 
de vídeo.

Uno de los más influyentes actualmente es el trabajo \textit{Investigating the Effectiveness 
of Cross-Attention to Unlock Zero-Shot Editing of Text-to-Video Diffusion Models}, analizando el papel 
de las capas de atención cruzada (véase Figura~\ref{fig:atencion-cruzada}) en modelos de difusión-texto-a-video (T2V) y estudia si las 
técnicas de edición desarrolladas previamente en el dominio texto-a-imagen pueden transladarse al formato 
de vídeo sin necesidad de reentrenamiento de los modelos. 

El atrículo divulgativo parte de la idea de que, en modelos texto-a-imagen, la atención cruzada desempeña 
un papel fundamental en la determinación de la forma, el tamaño y la localización de los objetos, además de 
garantizar coherencia en los prompts textuales. A partir de esa base, los autores investigan si la 
manipulación de mapas de atención cruzada (véase Figura~\ref{fig:atencion-cruzada}) en modelos T2V permiten controlar atributos 
como el tamaño, la posición o el movimiento de un objeto dentro del vídeo generado.

Se exploran diferentes estrategias de entrenamiento como la \textbf{orientación futura} y la 
\textbf{orientación pasada} (véase Figura~\ref{fig:orientación pasada}), permitiendo mediante instrucciones 
forzar el ajuste de un objeto en una imagen o el movimiento del mismo, modificando la atención sin alterar la arquitectura del modelo 
(más información), concluyendo el estudio en que la atención cruzada 
consitituye un mecanismo clave, habilitando la capacidad de un \textbf{modelo texto-a-video} para editar texto, imágenes 
o vídeos siguiendo instrucciones (\textit{prompts}) directod, sin haber sido entrenado específicamente 
para ello. El trabajo sienta así las bases para futuras investigaciones orientadas a mejorar la 
precisión de los mapas de atención y ampliar las capacidades de edición controlada en generación 
de vídeo mediante difusión.

\begin{figure}[H]
    \centering
    \includegraphics[width=1\textwidth]{assets/atencion-cruzada.png}
    \caption{Representación visual del concepto \textbf{atención cruzada}}
    \label{fig:atencion-cruzada}
\end{figure}

\begin{figure}[H]
    \centering
    \begin{tikzpicture}
    \node[draw=pim, line width=1pt, inner sep=1pt]
    {\includegraphics[width=1\textwidth]{assets/orientacion-pasada.png}};
    \end{tikzpicture}
    \caption{Representación visual del concepto \textbf{orientación pasada}}
    \label{fig:orientación pasada}
\end{figure}

\subsection{Fuentes}
\begin{itemize}

\item Brown, B. (2016). \textit{Cinematography: Theory and Practice: Image \item Making for Cinematographers and Directors. Focal Press. }
\item Dancyger, K. (2014). \textit{The Technique of Film and Video Editing: \item History, Theory, and Practice. Focal Press.}
\item Jack, K. (2019). \textit{Video Demystified: A Handbook for the Digital Engineer. Newnes. }
\item Watkinson, J. (2018). \textit{The Art of Digital Video. Focal Press. }
\end{itemize}



