\section{Estado del arte}

La creación y edición de vídeo constituye uno de los pilares fundamentales dentro del ámbito del Procesamiento de la Información Multimedia, ya que combina aspectos técnicos, artísticos y comunicativos. El desarrollo de un proyecto audiovisual eficaz requiere un conocimiento previo de los conceptos básicos que intervienen en la captura, tratamiento, codificación y distribución del contenido visual en movimiento. 

\subsection{Conceptos básicos}

Para hablar de la edición de vídeo digital debemos de tener claros algunos de sus elementos esenciales. Como el \textbf{píxel}, la unidad mínima de información visual de la que se compone una imagen. La disposición y cantidad de píxeles determinan la \textbf{resolución}, un factor que determina la calidad del vídeo. Entre las resoluciones más habituales se encuentran HD (1280×720), Full HD (1920×1080) y 4K (3840×2160), ampliamente utilizadas en distintos contextos audiovisuales según las necesidades de calidad y rendimiento (Jack, 2019). 

Otros conceptos fundamentales son la \textbf{relación de aspecto}, que define la proporción entre el ancho y el alto de la imagen, 16:9, estándar en televisión y plataformas digitales, y 21:9 en producciones cinematográficas. O la \textbf{fluidez del movimiento}, el número de fotogramas por segundo (FPS). Valores como 24 FPS se asocian tradicionalmente al cine, mientras que 30 o 60 FPS son habituales en contenidos televisivos y digitales, especialmente en videojuegos o retransmisiones en directo. Normalmente, cuanto mayor sea la resolución y tasa de fotogramas de un vídeo, mayor será la calidad de este, así como el tamaño de este.  

 Por otro lado, la \textbf{postproducción} es una de las fases más determinantes en la creación y edición de vídeo, ya que en ella se da forma definitiva al material audiovisual grabado. Las posibilidades de la edición son infinitas y van únicamente sujetas a la creatividad y conocimientos del editor. Uno de los principales objetivos es conseguir obtener una narrativa mezclando las distintas vistas (Dancyger, 2014). La postproducción incluye también la incorporación de efectos visuales (VFX), como transiciones, composiciones digitales, títulos o animaciones, que amplían las posibilidades creativas del medio y facilitan la integración de elementos gráficos en el vídeo final (Watkinson, 2018). 
 
 Paralelamente, el tratamiento del \textbf{audio}, que engloba procesos como la eliminación de ruido, la ecualización, la mezcla de pistas y la sincronización con la imagen, resulta esencial para garantizar una experiencia audiovisual inmersiva y profesional, siendo un componente clave en la percepción de calidad por parte del espectador. En conjunto, la postproducción actúa como un proceso integrador en el que convergen aspectos técnicos y artísticos, permitiendo transformar el material audiovisual bruto en un producto final que cumpla con los estándares de calidad exigidos en el ámbito del procesamiento multimedia. 

 \subsection{Fuentes}
\begin{itemize}

\item Brown, B. (2016). \textit{Cinematography: Theory and Practice: Image \item Making for Cinematographers and Directors. Focal Press. }
\item Dancyger, K. (2014). \textit{The Technique of Film and Video Editing: \item History, Theory, and Practice. Focal Press.}
\item Jack, K. (2019). \textit{Video Demystified: A Handbook for the Digital Engineer. Newnes. }
\item Watkinson, J. (2018). \textit{The Art of Digital Video. Focal Press. }
\end{itemize}



